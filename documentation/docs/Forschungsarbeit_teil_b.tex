\documentclass[12pt, a4paper, oneside]{article}
%die Pakete werden hier durch den Include-Befehl separat eingelesen
\include{Pakete}

\lstset{language=R}
\lstset{basicstyle=\ttfamily,
basicstyle=\small}
\lstset{literate=%
  {Ö}{{\"O}}1
  {Ä}{{\"A}}1
  {Ü}{{\"U}}1
  {ß}{{\ss}}1
  {ü}{{\"u}}1
  {ä}{{\"a}}1
  {ö}{{\"o}}1
}

\title{\textbf{Titel der Arbeit}}
\author{Miriam Lischke}

\setlength{\parindent}{0cm} %keine Einrückung
\linespread{1.5} 

\acsetup{first-style=short}
\newpage
\newpage
%Hier fängt die Deklarierung von Akronymen
%in alphabetischer Reihenfolge an
%auf Akronym im Text verweisen mittels \ac{ }
\DeclareAcronym{AIC}{
	short = AIC ,
	long  = \hspace*{22mm}Akaike-Informationskriterium ,
	tag = abbrev
}

\DeclareAcronym{BLES}{
	short = BLES ,
	long  = \hspace*{19mm}Bester Linearer Erwartungstreuer Schätzer ,
	tag = abbrev
}

\DeclareAcronym{c.p.}{
	short = c.p. ,
	long  = \hspace*{22mm}ceteris paribus ,
	tag = abbrev
}


\DeclareAcronym{u.i.v.}{
	short = u.i.v. ,
	long  = \hspace*{22mm}unabhängig und identisch verteilt ,
	tag = abbrev
}

\DeclareAcronym{KQS}{
	short = KQS ,
	long  = \hspace*{20mm}Kleinste-Quadrate-Schätzung ,
	tag = abbrev
}


\DeclareAcronym{VA}{
	short = VA ,
	long  = \hspace*{23mm}Varianzanalyse ,
	tag = abbrev
}


%\newpage
%\newpage
%in alphabetischer Reihenfolge

\newcounter{SeitenzahlSpeicher}
\begin{document}

\include{Titelseite}

\begin{spacing}{1}
\pagenumbering{Roman}
\setcounter{page}{2}
\tableofcontents
\end{spacing}
\newpage
\begin{spacing}{1}
\section*{Abbildungsverzeichnis} 
\addcontentsline{toc}{section}{Abbildungsverzeichnis}
\renewcommand{\listfigurename}{}
\listoffigures
\end{spacing}

\newpage
\clearpage
\setcounter{SeitenzahlSpeicher}{\value{page}}
\pagenumbering{arabic}
\newpage
\pagenumbering{arabic}

\include{einfuerung}
\include{kapitel-2}
\include{kapitel-3}
\include{kapitel-4}
\include{kapitel-5}
\newpage
\clearpage
\pagenumbering{Roman}
\setcounter{page}{\theSeitenzahlSpeicher}
%\blindtext[3]


%mit Bibtex gearbeitet:
%Falls die Referenzen nicht funktionieren, mehrmals kompilieren mit F6 - F11 - F6 - F6 und dann F1 (Reihenfolge beachten) versuchen
%google scolar dann auf zitieren Bixtex klicken und copy paste
%thebibliography=Datei auf die Zugegriffen wird
%\large If you want to learn about linear and semiparametric regression, you should try \cite{Regression}, whereas \cite{Hobbit} will be disappointing in this context.


\section{Literaturverzeichnis}
\bibliographystyle{gerplain}
\renewcommand{\refname}{} 
\bibliography{bibba}

\newpage


\appendix %Anhang
\pagenumbering{arabic}
%

%
\newpage

\end{document}
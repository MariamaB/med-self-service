\section{Kritische Betrachtung und Limitationen}
Die Bereitstellung eines Self-Service zur Überprüfung der eigenen zugrundeliegenden digitalen Gesundheitsanwendung ist eine praktische und übersichtliche Weise für Hersteller vor Antragsstellung eine Einschätzung einzuholen. So erhalten Sie in einer kompakten Form, die Auskunft ob sich der gesamte Aufwand lohnt. Das spart Zeit als auch Kosten, was vor allem kleinere Unternehmen sowie Startups sehr entgegen kommt. Allerdings handelt es sich hier nach erfolgreicher Bearbeitung des Canvas zu nächst nur um eine Momentaufnahme. Der Canvas verändert sich nicht mit den Fortschritten der App dynamisch mit. Interessant wäre es, wenn es sich bei dem Self-Service um eine Plattform handeln würde, auf der man sein Projekt zur DiGA verwaltet. Und somit die Hersteller bis zur Antragsstellung begleitet während man weitere Fortschritte dokumentieren kann. Zudem könnte diese Plattform während jeder Phase weitere Tipps und Tricks zur Verfügung stellen, die dabei helfen schneller das gewünschte Ziel zu erreichen.
Ein anderer Aspekt, der in diesem Zusammenhang noch nicht betrachtet wird, ist eine Schnittstelle zum Antragsverfahren der BfArm selbst. Zum Anstoßen des Prüfungsverfahren müssen vorgegebene Formulare eingereicht werden. Da es sich bei denen überwiegend um die selben Informationen wie auch im Canvas handelt, wäre es noch effektiver, wenn der Self-Service zusätzliche diese Formulare im Hintergrund ausfüllt. Diese könnten dann dem Hersteller für weitere Schritte zur Verfügung gestellt werden oder der Antrag vollständig über diese Plattform vorbereitet und eingereicht werden.
\section{Fazit und Ausblick}
Das Ziel, das zur Anfang in Form von Forschungsfragen formuliert wurde, wurde weitgehend erreicht. Einen digitaler Canvas, gebettet in einem einfachen Self-Service, der alle Informationen zur Überprüfung einer digitalen Gesundheitsanwendungen auf Erstattungsfähigkeit enthält, wurde umgesetzt. Der Nutzer dieses Self-Service kann bereits eine erste Version des Canvas mit den eingegebenen Daten als pdf downloaden, jedoch ist das System an dieser Stelle noch nicht soweit eine Auswertung nach DIGAV Verordnung vorzunehmen. Einen Ausblick wäre es diese Funktionalität weiter auszubauen und die Qualität der generierten PDF zu optimieren.

\section{Einführung}
\subsection{Situationsbeschreibung}
\emph{Im Zuge der zunehmenden Digitalisierung} des Gesundheitswesens, ergeben sich für IT spezialisierte bzw. Unternehmen im Allgemeinen vermehrt Perspektiven in der Gesundheitsbranche fußzufassen. 
Aus diesen neuen Möglichkeiten der Branche entstehen eine Vielzahl an Kooperationen zwischen Experten aus verschiedensten Bereiche, wie etwa: Forschung, Wissenschaft, Public Health und Technologie etc. Dabei entstehen innovative Produkte, die dazu beitragen, das Gesundheitswesen voranzubringen. ``Mit dem Inkrafttreten des Digitale-Versorgung-Gesetzes (DVG) am 19. Dezember 2019 wurde die „App auf Rezept“ für Patientinnen und Patienten in die Gesundheitsversorgung eingeführt. 
Damit haben ca. 73 Millionen Versicherte in der gesetzlichen Krankenversicherung (GKV) einen Anspruch auf eine Versorgung mit digitalen Gesundheitsanwendungen (DiGA), die von Ärzten und Psychotherapeuten verordnet werden können und durch die Krankenkasse erstattet werden.''~\cite{leitfadenfasttrack}  
Dieser neue gesetzliche Rahmen schafft eine enorme Geschäftsnische für Unternehmen. 

Jedoch um für ein Produkt den offiziellen Status einer DiGA\footnote{Digitale Gesundheitsanwendung} zu erhalten muss mittels eines aufwendigen Prüfungsverfahren durch das Bundesinstitut für Arzneimittel und Medizinprodukte (BfArM), nach §139e SGB V ermittelt werden ob alle Voraussetzungen erfüllt sind. Erst nach erfolgreicher Prüfung erfolgt die Aufnahme in ein Verzeichnis für erstattungsfähige digitale Gesundheitsanwendungen (\textit{DiGA-Verzeichnis}~\cite{digaverzeichnis}) des BfArM\footnote{Bundesinstitut für Arzneimittel und Medizinprodukte\label{ftn:bfarm}}.

\subsection{Motivation und Forschungsfragen}
Das Verfahren nachdem das BfArM die potenzielle digitale Gesundheitsanwendung prüft nennt sich Fast-Track-Verfahren und wird durch einen Antrag zur Aufnahme in das DiGA-Verzeichnis angestoßen. Obwohl das Verfahren ``als zügiger ``Fast-Track'' (=schneller Weg) konzipiert''~\cite{digahersteller} ist, beträgt die Bearbeitungszeit drei Monate nach Eingang des vollständigen Antrags.

Zur Orientierung stellt das BfArM, den Antragssteller einen Umfangreichen Leitfaden gemäß § 139e Absatz 8 Satz 1 SGB V zur Verfügung. Dieser wendet sich in erster Linie an die Hersteller, denn die Anforderungen sind sehr hoch.

Der Leitfaden soll das Antragsverfahren in seinen Schritten übersichtlich darstellen und schafft Transparenz über die konkret zu erfüllenden Anforderungen im Verfahren und stellt sicher, dass alle Anträge nach gleichen Maßstäben bearbeitet werden.

Da der Leitfaden eher eine zusammenfassende Darstellung der
Regelungen, die an verschiedenen Stellen im SGB V\footnote{Sozialgesetzbuch, Fünftes Buch - Gesetzliche Krankenversicherung\label{ftn:sgb}}, in der DiGAV\footnote{Digitale-Gesundheitsanwendung-Verordnung\label{ftn:digav}} als auch in den Anlagen zur DiGAV zu finden sind, leistet und somit nur darlegt wie es die normativen Vorgaben aus DVG\footnote{Digitale-Versorgung-Gesetz\label{ftn:dvg}} und DiGAV regelmäßig auslegen.
Können vor allem junge Unternehmen wie Startups schnell den Überblick verlieren ob sie bereits die Voraussetzungen zur Antragsstellung erfüllen oder welche Schritte vorher noch eingeleitet werden müssen. Dies fordert viel Vorbereitung, Geld und Zeit. Da wäre es vom großen Vorteil mittels eines Vorabchecks zu prüfen ob bereits alle Anforderungen für einen erfolgreichen Ausgang der Antragsstellungen erfüllt sind oder in Form einer Roadmap aufweist welche Schritte dazu noch erforderlich sind. Daraus ergeben sich konkret die folgenden Forschungsfragen:
%Andererseits können all diese Anforderungen und zusammenfassenden Darstellungen der Reglungen, die unter anderem an verschiedenen Stellen im SGB V, in der DIGAV und in den Anlagen zur DIGAV zufinden sind vor allem junge Unternehmen wie Startups übervordern. 

%Zur Orientierung stellt das BfArM, den Antragssteller einen Umfangreichen Leitfaden gemäß § 139e Absatz 8 Satz 1 SGB V zur Verfügung, denn die Anforderungen an die Hersteller sind sehr hoch.
\begin{enumerate}
         \item Wie lässt sich für einer MedTech-Hersteller ein vereinfachter, digitaler Prozess zur Selbsteinschätzung der DiGA-Konformität seiner Software-Applikationen gestalten?
         %Lasst sich mittels eines entwickelten digitalen Self-Service (sog. "Fast-Track-Checker"), die Vorbereitung zur Antragsstellung auf Aufnahme einer DIGA in das Verzeichnis, die durch das Fast-Track-Verfahren geprüft wird, für Antragssteller bzw. Hersteller vereinfachen?
         \item Welche technische Realisierungsmöglichkeit ist geeignet, um einen derartigen Self-Service zur Vorbereitung des DiGA-Herstellers für die Antragstellung zum DiGA-Fast-Track-Verfahren zu ermöglichen?
         %der durch den Self-Service entstehenden DiGA-Canvas, zu einer druckfähigen PDF generiert werden, sodass er für Hersteller zur Übersicht der aktuellen Situation und Hilfestellung für die folgenden Schritte dient?
\end{enumerate}

\subsection{Vorgehensweise}
Ziel dieser Forschungsarbeit ist es ein digitalen Self-Service zu entwickeln, der bei der Vorbereitung zum Antrag auf Aufnahme in das DiGA-Verzeichnis unterstützt und mithilfe eines DiGA-Canvas, die Anforderungen und Voraussetzungen kompakt zusammenfässt. Dabei wird ausgiebig erforscht, wie sich die Prüfung durch das Fast-Track-Verfahren zusammensetzt und welche Anforderungen sich dabei an die Hersteller und die Anwendung richten. Des Weiteren wird erforscht welche relevanten Informationen auf dem DiGA-Canvas zur Übersicht zusammengefasst werden sollten und mithilfe welcher Technologie, eine druckfähige PDF generiert werden kann.